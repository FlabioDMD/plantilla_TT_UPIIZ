\chapter{Antecedentes}\label{cap2}
Es el primer cap�tulo del Reporte Final y aporta los elementos te�ricos b�sicos que justifican y dan referencia a la metodolog�a de trabajo empleada para resolver el problema, dependiendo del trabajo que se est� realizando, este cap�tulo se llamar�: marco te�rico o, marco conceptual.


\begin{figure}[H]
	\centering
	\includegraphics[scale=1]{upiiz}
	\caption{Este es un t�tulo muy largo para una Figura.}
	\label{fig2}
\end{figure}


\begin{equation}\label{key}
\left [
\begin{array}{c}
\dot{x}\\
\dot{y} \\
\dot{\theta}
\end{array}
\right ]
\end{equation}

\begin{align}\label{key}
\sum_{i}^{m}&=0\\
\sum&=10
\end{align}



\begin{table}[H]	
	\centering
	\caption{Tabla demostrativa 2.}
	\begin{tabular}{|c|c|c|}
		\hline
		Celda 1,1&Celda 1,2&Celda 1,3\\\hline
		Celda 2,1&Celda 2,2&Celda 2,3\\\hline
	\end{tabular}
\end{table}